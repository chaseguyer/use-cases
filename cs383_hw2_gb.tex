\documentclass[12pt]{report}
\usepackage{geometry}
\usepackage{titling}

\newcommand{\subtitle}[1]{
	\posttitle{
		\par\end{center}
		\begin{center}\large#1\end{center}
		\vskip0.5em}
}

\newlength\tindent
\setlength{\tindent}{\parindent}
\setlength{\parindent}{0pt}
\renewcommand{\indent}{\hspace*{\tindent}}

\geometry{margin=1in}

\setcounter{tocdepth}{2}

\title{Homework 1}
\subtitle{CS 383 - Group 3}
\author{
Mason Fabel \\
\and Anton Stalick \\
% TODO: Add your names here.
}
\date{\today}


\begin{document}

\maketitle

\tableofcontents
Document Typsetting and Editing: \textbf{Mason Fabel}
\clearpage

\chapter{Individual Use Cases}

% TODO: Copy and paste my section to do your individual use cases.

% TODO: Give each SUBSECTION a descriptive title.

% TODO: Give each USE CASE a USE CASE DIAGRAM in the form of a .png image
% generated with PlantUML. Make sure the PlantUML file is committed into
% this repository.

% NOTE: Adding a .png can be done with \includegraphics{foo.png} where 
% foo.png is in the same directory as this .tex

% NOTE: LaTeX has a `\begin{subsubsection}{title}` if you desire more
% granular sectioning.

% NOTE: I disabled the automatic indentation of new paragraphs, as it was
% causing some strange document layouts. If you want something indented,
% then just use the \indent command. --Mason

\begin{section}{Mason Fabel}
\begin{subsection}{Interact with a Coworker}
\textbf{Actors}:
Player, NPC

\textbf{Goal}:
The Player wishes to perform an interaction with an NPC.

\textbf{Preconditions}:
\begin{enumerate}
\item The Player is adjacent to the NPC.
\item The Player has a relationship value with the NPC equal to or greater
	than the NPC's relationship threshold.
\end{enumerate}

\textbf{Summary}:
The Player interacts with an NPC occupying space in the World. The
interaction may or may not cause a state change for either the Player or
the NPC Social Network.

\textbf{Related Use Cases}:
All specific NPC interactions extend this use case. Examples include
modifying a relationship, gaining a promotion, or managing quests.

\textbf{Steps}:
\begin{enumerate}
\item The Player inputs the interact command, selecting the NPC.
\item The System performs the interaction, gathing aditional input from the
	Player as necessary.
\end{enumerate}

\textbf{Alternative}:
The NPC informs the Player they lack the relationship to perform the
interaction (see \textit{Preconditions}).
\end{subsection}

\begin{subsection}{Get a Promotion}
\textbf{Actors}:
Player, NPC

\textbf{Goal}:
The Player seeks to advance to the next level of play.

\textbf{Preconditions}:
\begin{enumerate}
\item The Player is able to perform the use case \textit{Interact with a
	Coworker}.
\item The Player has recieved a Level Advancement Quest from the NPC.
\item The Player has completed the Level Advancement Quest given by the
	NPC.
\end{enumerate}

\textbf{Summary}:
The Player interacts with an NPC occupying space in the World. If all of
the \textit{Preconditions} are met then the Player is given access to the
next level of play, and is given any corresponding benefits or items.

\textbf{Related Use Cases}:
This use case extends \textit{Interact with a Coworker}. This case may be
optionally extended to allow for different player classes or alternate
advancement requirements.

\textbf{Steps}:
\begin{enumerate}
\item The Player inputs the interact command, selecting the NPC.
\item The System enables the next level of play.
\item The System enacts any appropriate state changes upon the Player.
\item The System enacts any appropriate state changes upon NPCs.
\item The Player is notified of their advancement.
\end{enumerate}

\textbf{Alternative}:
The NPC notifies the Player of the conditions that remain unmet (see
\textit{Preconditions}).
\end{subsection}

\begin{subsection}{Combat}
\textbf{Actors}:
Player(s), Enemy Entity(-ies)

\textbf{Goal}:
The Player seeks to overcome and defeat an opponent or challenge.

\textbf{Preconditions}:
\begin{enumerate}
\item The Player is adjacent to the Enemy Entity.
\item The Player has the Combat flag enabled.
\item The Enemy Entity has the Combat flag enabled.
\end{enumerate}

% NOTE: Whoever does the Player Victory and Player Death use cases needs
% to make sure the System enables the Combat flag on the Player.

\textbf{Summary}:
The Player engages in combat with an Enemy Entity occupying space in the
World. Combat is carried out in Turns, with each combatant selecting a Move
each Turn, which are then resolved in order from the entity with the
highest Speed value to the entity with the lowest Speed value. When every
member of one side has their Health reduced to 0, then the other side is
victorious. If the Player(s) won, then go to use case \textit{Player
Victory}. If the Player(s) lost, then go to use case \textit{Player Death}.

\textbf{Related Use Cases}:
This use case is a parent to \textit{Combat Turn}. The use cases
\textit{Player Victory} or \textit{Player Death} immediately follow. This
use case can be optionally extended to allow for alternate forms of combat.

\textbf{Steps}:
\begin{enumerate}
\item The Player inputs the Begin Combat command, \textit{or} the Enemy
Entity begins combat with the Player.
\item The System disables the Player's Combat flag.
\item The System disables the Enemy Entity's Combat flag.
\item \label{combat:turn} Perform use case \textit{Combat Turn}.
\item If neither side has been reduced to 0 Health,
	go to step \ref{combat:turn}.
\item If the Enemy Entity has a Health of 0, go to use case \textit{Player
	Victory}.
\item If the Player has a Health of 0, go to use case \textit{Player
	Death}.
\end{enumerate}
\end{subsection}
\end{section}

\begin{section}{Anton Stalick}
\begin{subsection}{World Entity Interaction}

\textbf{Actors}:
Player, Interactable Entity

\textbf{Preconditions}:
\begin{enumerate}
\item Player is adjacent to the Interactable Entity.
\item Player inputs the interact command, selecting the Interactable Entity.
\end{enumerate}

\textbf{Summary}:
The Player interacts with the Entity occupying space in the World. The Entity 
produces its interactive effect. 

\textbf{Related Use Cases}
All World Entity Interactions extend this use case. Examples would be 
opening a door or operating a switch.

\textbf{Steps}:
\begin{enumerate}
\item Player moves adjacent to an Interactable Entity.
\item System enables the input of the interact command.
\item Player inputs the interact command.
\item System evaluates the effect of the interact command on the
Interactable Entity and updates the Entity's state accordingly.
\item Player moves away from the Interactable Entity.
\item System disables the input of the interact command.
\end{enumerate}

\end{subsection}

\begin{subsection}{Inventory Item Use}

\textbf{Actors}:
Player, Inventory, Useable Item

\textbf{Preconditions}:
\begin{enumerate}
\item Inventory belongs to Player.
\item Item is in Inventory.
\item Item is useable from the Inventory
\item Player inputs the use command, selecting Item for use.
\end{enumerate}

\textbf{Summary}:
The Player uses the Item in their Inventory. The Item produces its
use effect.

\textbf{Related Use Cases}
All Useable Item Uses extend this use case. Examples would be quaffing
a potion or reading a note.

\textbf{Steps}:
\begin{enumerate}
\item Player selects Item from Inventory.
\item System enables the input of the use command.
\item Player inputs the use command.
\item System evaluates the effect of the use command on the
Useable Item and updates the Item's state accordingly.
\end{enumerate}

\end{subsection}
\end{section}

\begin{section}{Foo Bar}
\begin{subsection}{Use Case 1}
\textbf{Actors}:

\textbf{Goal}:

\textbf{Preconditions}:

\textbf{Summary}:

\textbf{Related Use Cases}:

\textbf{Steps}:

\textbf{Alternative}:
\end{subsection}

\begin{subsection}{Use Case 2}
\textbf{Actors}:

\textbf{Goal}:

\textbf{Preconditions}:

\textbf{Summary}:

\textbf{Related Use Cases}:

\textbf{Steps}:

\textbf{Alternative}:
\end{subsection}
\end{section}

\chapter{Team Use Cases}

% TODO: Meet FRIDAY after class to select use cases for this section

\end{document}
